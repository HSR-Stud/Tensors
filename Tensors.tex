

\documentclass[a4paper]{scrartcl}

% Mathematik-Pakete
\usepackage{amsmath}
\usepackage{amssymb}
\usepackage{amstext}
\usepackage{amsfonts}
\usepackage{mathrsfs}
\usepackage{paralist}
\usepackage[pdftex]{graphicx}
\usepackage{fancyhdr}

%\usepackage[ansinew]{inputenc}
%\usepackage[T1]{fontenc}

\usepackage[utf8]{inputenc}
         
\usepackage[a4paper]{geometry}
\geometry{verbose,tmargin=1.5cm,lmargin=2cm,rmargin=2cm,bmargin=3cm,nohead}
\usepackage[ngerman]{babel}


\begin{document}
\pagestyle{fancy}
%\setlength{\footskip}{5mm}
\fancyhf{}
\fancyfoot[C]{\thepage} %Seitennummer
\renewcommand{\headrulewidth}{0pt}
\renewcommand{\footrulewidth}{0.5pt}

%\twocolumn[
 \centerline{\LARGE \bf \textsf{MSE CFD}} 
 \smallskip
\centerline{\Large \bf \textsf {Zusammenfassung}}
\medskip
  \centerline{\bf \textsf{dstrebel, s1bischo, sboller }}

 \smallskip \noindent\rule{\textwidth}{0.5pt}
\smallskip%]



\section{Juhui Tensoren...}

Gradient eines Skalarfeldes: gibt die Richtung und Stärke des steilsten Anstiegs
eines Skalarfeldes an.
Skalarfeld => Vektorfeld

Divergenz eines Vektorfeldes (Quellendichte eins Vektorfeldes): Gibt die Tendenz
eines Vektorfeldes an, von Punkten wegzufliessen (das gilt für positives
Vorzeichen; bei negativem Vorzeichen handelt es sich dementsprechend um die
Tendenz zu den Punkten hinzufliessen) Vektorfeld => Skalarfeld

Rotation eines Vektorfeldes: Gibt die Tendenz eines Vektorfeldes an, um Punkte
zu rotieren; die Rotation eines Vektorfeldes ist ein Vektorfeld von
Pseudovektoren. Und zwar folgt aus dem Stokes’schen Integralsatz (siehe unten),
dass die Rotation die lokale Wirbeldichte eines Vektorfeldes beschreibt.

Quelle: http://de.wikipedia.org/wiki/Vektoranalysis



\section{Description of Physical Phenomena}
\subsection{Einheiten}

$ [\rho]=\frac{kg}{m^3} (Dichte)$\\
$ [c_p]=\frac{J}{kgK} $\\
$ [m]=kg (Masse)$\\
$ [T]=K (Temperatur)$\\
$ [j_w]=\frac{W}{m^2} (Energieflussdichte)$\\
$ [\lambda]=\frac{W}{mK} $\\
$ [\nabla]=\frac{1}{m} $\\
$ [Q]=As (el. Ladung)$\\
$ [F]=N (Kraft)$\\
$ [W]=W (thermische Energie)$\\
$ [w]=\frac{kg}{ms^2} nochmals kontrollieren!!! $\\

\subsection{SI-Einheiten}
$ J = \frac{kg m^2}{s^2} $\\

\subsection{Begriffe}
\begin{itemize}
\item extensive Grösse: wenn sich System verdoppelt verdoppelt sich auch die Eigenschaft z.B. Impuls, Volumen ...
\item intensive Grösse: wenn sich System verdoppelt verdoppelt sich Eigenschaft nicht z.B. Druck, Temperatur, ...
\end{itemize}


\section{Bilanzgleichungen}

\begin{tabular}{|c|c|c|c|c|}
\hline  & Stoffliche Grösse & Potential & Balance Equation uniform & Balance Equation distributed \\ 
\hline Charge Transport & Charge Q & el. Potential ??? & $\frac{dQ}{dt}=I_Q$ & $ \frac{\parallel p_Q}{\partial t}=-\nabla\cdot j_Q$ \\ 
\hline Heat Transport & thermal Energy W & Temperatur T & $\frac{\parallel W}{\partial t}= I_W+Q_W$ & $\frac{\parallel w}{\partial t}= -\nabla \cdot j_W+q_W$  \\ 
\hline Particle Transport & Partical number N & Particle density c & $\frac{dN}{dt}=I_N+Q_N$ & $\frac{dp}{dt}=-\nabla \cdot j_W +q_W$ \\ 
\hline Mechanics (Momentum Transport) & Momentum p & Velocity v  & $\frac{\partial p}{\partial t}=F$ & ?? \\ 
\hline 
\end{tabular} 

\section{Constitutive Laws}

\begin{tabular}{|c|c|c|c|}
\hline  & Storage Laws & Constitutive Laws & Source Rates and Sourcerate Densities \\
\hline Charge Transport & $j_Q=-\sigma \cdot \nabla ??? = \sigma \cdot E$ & & \\
 
\hline Heat Transport & $w=c_p \rho T$ & $j_W=-\lambda \cdot \nabla T$ &  \\ 
\hline Particle Transport (Diffusion) & c=c & $j_N=- D \cdot \nabla c_N$ &  \\ 
\hline Mechanics & $p=mv$ & $F_hook = - k \Delta u$ (Hook's Law) $F_visc=-C \eta \cdot v$ (Viscous Drag) &  \\ 
\hline 
\end{tabular} 




\section{Vektoranalysis und Integraltheorem}
\subsection{Nabla Operator}

\begin{align}
\nabla \: \mathrm{oder} \: \vec \nabla = \left (\frac\partial{\partial
x_1},\ldots, \frac\partial{\partial x_n}\right)
\end{align}
\subsection{Gradient}
Der Gradient ist ein mathematischer Operator, genauer ein Differentialoperator,
der auf ein Skalarfeld angewandt werden kann und in diesem Fall ein
Gradientenfeld genanntes Vektorfeld liefert, das die Änderungsrate d Richtung
der größten Änderung des Skalarfeldes angibt.

\begin{align}
\operatorname{grad}(f(x,y,z)) = \nabla f = \frac{{\partial f}}{{\partial x}} +
\frac{{\partial f}}{{\partial y}} + \frac{{\partial f}}{{\partial z}} 
\end{align}
\subsection{Divergenz}
In der Mathematik ist die Divergenz ein Differentialoperator, der einem
Vektorfeld ein Skalarfeld zuordnet. Während bei einem Vektorfeld jedem Punkt ein
Vektor zugeordnet wird, wird bei einem Skalarfeld jedem Punkt ein Skalar, also
eine Zahl, zugeodnet. Interpretiert man das Vektorfeld als Strömungsfeld, so
gibt die Divergenz für jeden Raumpunkt an, wie viel mehr aus einer Umgebung
dieses Punkts hinausfließt als in sie hineinfließt. Mithilfe der Divergenz lässt
sich also herausfinden, ob und wo das Vektorfeld Quellen (Divergenz größer als
Null) oder Senken (Divergenz kleiner als Null) hat. Ist die Divergenz überall
gleich Null, so bezeichnet man das Feld als quellenfrei.

\begin{align}
\vec F = \begin{pmatrix} F_1 \\ F_2 \\ F_3 \end{pmatrix}
\end{align}

\begin{align}
\operatorname{div}\vec F = \nabla \cdot \vec F =
\frac{\partial}{\partial x_1}F_1 + \frac{\partial}{\partial x_2}F_2+ \frac{\partial}{\partial x_3}F_3
\end{align}

\subsection{Rotation}
Als Rotation bezeichnet man in der Vektoranalysis, einem Teilgebiet der
Mathematik, einen bestimmten Differentialoperator, der einem Vektorfeld im
dreidimensionalen euklidischen Raum mit Hilfe der Differentiation ein neues
Vektorfeld zuordnet. Handelt es sich beispielsweise um ein Strömungsfeld, so
gibt die Rotation für jeden Ort das Doppelte der Winkelgeschwindigkeit an, mit
der ein mitschwimmender Körper rotiert, also wie schnell und um welche Achse er
sich dreht. Dieser Zusammenhang ist namensgebend.

\begin{align}
\mathbf{\operatorname{rot}}\,\vec F(x,y,z) = \nabla\times \vec F =
\begin{pmatrix} \frac{\partial}{\partial x} \\ \frac{\partial}{\partial y} \\
\frac{\partial}{\partial z} \end{pmatrix} \times \begin{pmatrix} F_x\\ F_y\\ F_z
\end{pmatrix} = \begin{pmatrix} \frac{\partial F_z}{\partial y} - \frac{\partial
F_y}{\partial z} \\ \frac{\partial F_x}{\partial z} - \frac{\partial
F_z}{\partial x} \\ \frac{\partial F_y}{\partial x} - \frac{\partial
F_x}{\partial y} \end{pmatrix}
\end{align}

\subsection{Rechenregeln}
Es seien $\vec F$, $\vec G$ Vektorfelder und $U$, $V$ Skalarfelder.

\begin{align}
\operatorname{rot}(\operatorname{grad}U)=\nabla \times (\nabla U) = 0
\end{align}

\begin{align}
\operatorname{div}(\operatorname{rot}\vec{F}) = \nabla \cdot (\nabla \times
\vec F) = 0
\end{align}

\begin{align}
\operatorname{rot}(\operatorname{rot}\vec{F}) =
\operatorname{grad}(\operatorname{div}\vec{F}) -\Delta \vec{F}
\end{align}

\begin{align}
\operatorname{grad}(UV)=U\operatorname{grad}V+V\operatorname{grad}U
\end{align}

\begin{align}
\operatorname{grad}(\vec{F}\cdot \vec{G}) =
(\operatorname{grad}\vec{F})^{\operatorname t}\vec{G} + (\operatorname{grad}\vec{G})^{\operatorname t}\vec{F}
\end{align}

\begin{align}
\operatorname{div}(U\vec{F})=U\operatorname{div}\vec{F}+
\vec{F}\cdot\operatorname{grad}U
\end{align}

\begin{align}
\operatorname{div}(\vec{F}\times \vec{G})= \vec{G} \cdot
\operatorname{rot}\vec{F} -\vec{F}\cdot\operatorname{rot}\vec{G}
\end{align}

\begin{align}
\operatorname{rot}(U\vec{F})= U\operatorname{rot}\vec{F}
-\vec{F}\times\operatorname{grad}U\
\end{align}
\subsection{Satz von Gauss}
Der Integralsatz von  liefert den Zusammenhang zwischen einem Volumenintegral
über ein Volumen $V$, das von einem Feld $\vec F$ durchsetzt ist, und einem
Oberflächenintegral über die dieses Volumen umschliessende Fläche $A$. Die
Orientierung der Fläche sei so festgelegt, daß die Aussenseite die positive
Seite ist.

\begin{align}
\int_V \operatorname{div} \vec u \; \mathrm d^{(n)}V = \oint_{S} \vec u \cdot
\vec n\; \mathrm d^{(n-1)}A\,.
\end{align}
\subsection{Satz von Stokes}
Das Kurven- oder Linienintegral eines räumlichen Vektorfeldes $\vec u$ längs
einer geschlossenen Kurve $\partial F$ ist gleich dem Oberflächenintegral der
Rotation von $\vec u$ über eine beliebige Fläche $F$, die durch die Kurve
$\partial F$ berandet wird:

\begin{align}
 \int_{F} \operatorname{rot} \vec u \cdot d \vec A =
\oint_{\partial F} \vec u \cdot d \vec r
\end{align}


\section{Mathematical Basics}
\subsection{Grundlagen}


Einsteinnotation: $w_i=\sum_{j=1}^{3}A_ijv_j=A_ijv_j$
\\
Orthogonale Matrix ist, wenn gilt $MM^T=E$
\\


\subsection{Vector Algebra}
\begin{tabular}{|c|c|c|}
\hline  & Allgemein & in components \\ 
\hline Vector Addition & $w=u+v$ & $w_i=u_i+v_i$ \\ 
\hline Multiplication with a Scalar & $w=\alpha v$ & $w_i=\alpha v_i$ \\ 
\hline Scalar Product & $\alpha = v \cdot w $ & $\alpha = v_i w_i$ \\ 
\hline 
\end{tabular} 

\subsection{Tensor Algebra}
\begin{tabular}{|c|c|c|}
\hline  & Allgemein & in components \\ 
\hline Addition & $C=A+B$ & $C_{ij}=A_{ij}+v_{ij}$ \\ 
\hline Multiplication& $B=\alpha A$ & $B_{ij}=\alpha A_{ij}$ \\ 
\hline Product I & $C = a \otimes b $ & $C = a_i b_i$ \\ 
\hline Product II& $C = A \otimes B $ & $C = A_{ij} B_{ij}$ \\
\hline 
\end{tabular} 

\subsection{Tensorverjüngung}
\begin{tabular}{|c|c|c|}
\hline  & Allgemein & in components \\ 
\hline  & $c=A:B$ & $c=A_{ij}+v_{ij}$ \\ 
\hline  & $B=\mathbb{D}:A$ & $B_{kl}=D_{klij}+A_{ij}$ \\
\hline  & $C=AB$ & $C_{ij}=A_{ik}+B_{kj}$ \\
\hline Multiplication & $w=Av$ & $w_i=A_{ij}+v_{j}$ \\
\hline 
\end{tabular} 

\subsection{Eigenwerte}
$det(A-\lambda E)$


\subsection{Übungen}
$E'_{kl}=R_{ki}R_{lj}E_{ij}=R_{ki}E_{ij}R_{jl}^T$
\\
\\
Rotationsmatrix
\\
$R_z(\alpha) = \begin{pmatrix} 
\cos \alpha & -\sin \alpha & 0 \\
\sin \alpha &  \cos \alpha & 0 \\            
   0        &  0           & 1
\end{pmatrix}$
\\



\end{document}




