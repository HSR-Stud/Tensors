

\documentclass[a4paper]{scrartcl}

% Mathematik-Pakete
\usepackage{amsmath}
\usepackage{amssymb}
\usepackage{amstext}
\usepackage{amsfonts}
\usepackage{mathrsfs}
\usepackage{paralist}
\usepackage[pdftex]{graphicx}
\usepackage{fancyhdr}

%\usepackage[ansinew]{inputenc}
%\usepackage[T1]{fontenc}

\usepackage[utf8]{inputenc}
%\usepackage[latin1]{inputenc}  
%\usepackage[T1]{fontenc}
%\usepackage{lmodern}          
\usepackage[a4paper]{geometry}
\geometry{verbose,tmargin=1.5cm,lmargin=2cm,rmargin=2cm,bmargin=2cm,nohead}
\usepackage[ngerman]{babel}


\begin{document}
\pagestyle{fancy}
\setlength{\footskip}{10mm}
\fancyhf{}
\renewcommand{\headrulewidth}{0pt}
\renewcommand{\footrulewidth}{0.5pt}

%\twocolumn[
 \centerline{\LARGE \bf \textsf{MSE CFD}} 
 \smallskip
\centerline{\Large \bf \textsf {Zusammenfassung}}
\medskip
  \centerline{\bf \textsf{dstrebel, s1bischo, sboller }}

 \smallskip \noindent\rule{\textwidth}{0.5pt}
\smallskip%]



\section{Juhui Tensoren...}

Gradient eines Skalarfeldes: gibt die Richtung und Stärke des steilsten Anstiegs
eines Skalarfeldes an.
Skalarfeld => Vektorfeld

Divergenz eines Vektorfeldes (Quellendichte eins Vektorfeldes): Gibt die Tendenz
eines Vektorfeldes an, von Punkten wegzufliessen (das gilt für positives
Vorzeichen; bei negativem Vorzeichen handelt es sich dementsprechend um die
Tendenz zu den Punkten hinzufliessen) Vektorfeld => Skalarfeld

Rotation eines Vektorfeldes: Gibt die Tendenz eines Vektorfeldes an, um Punkte
zu rotieren; die Rotation eines Vektorfeldes ist ein Vektorfeld von
Pseudovektoren. Und zwar folgt aus dem Stokes’schen Integralsatz (siehe unten),
dass die Rotation die lokale Wirbeldichte eines Vektorfeldes beschreibt.

Quelle: http://de.wikipedia.org/wiki/Vektoranalysis



\section{Description of Physical Phenomena}
\subsection{Einheiten}

$ [\rho]=\frac{kg}{m^3} $\\
$ [c_p]=\frac{J}{kgK} $\\
$ [m]=kg $\\
$ [T]=K $\\

\subsection{SI-Einheiten}
$ J = \frac{XXX}{XXX} $\\

\subsection{Begriffe}
\begin{itemize}
\item extensive Grösse: wenn sich System verdoppelt verdoppelt sich auch die Eigenschaft z.B. Impuls, Volumen ...
\item intensive Grösse: wenn sich System verdoppelt verdoppelt sich Eigenschaft nicht z.B. Druck, Temperatur, ...
\end{itemize}



\section{Vektoranalysis und Integraltheorem}
\subsection{Satz von Gauss}
Der Integralsatz von  liefert den Zusammenhang zwischen einem 
Volumenintegral über ein Volumen $V$, das von einem Feld $\vec F$ durchsetzt
ist, und einem Oberflächenintegral über die dieses Volumen umschliessende
Fläche $S$. Die Orientierung der Fläche sei so festgelegt, daß die Aussenseite
die positive Seite ist.

\begin{align}
\int_V \operatorname{div} \vec F \; \mathrm d^{(n)}V = \oint_{S} \vec F \cdot
\vec n\; \mathrm d^{(n-1)}S\,.
\end{align}
\subsection{Satz von Stokes}
Das Kurven- oder Linienintegral eines räumlichen Vektorfeldes $\vec u$ längs
einer geschlossenen Kurve $\partial F$ ist gleich dem Oberflächenintegral der
Rotation von $\vec u$ über eine beliebige Fläche $F$, die durch die Kurve
$\partial F$ berandet wird:

\begin{align}
 \int_{F} \operatorname{rot} \vec u \cdot d \vec A =
\oint_{\partial F} \vec u \cdot d \vec r, 
\end{align}
\end{document}




