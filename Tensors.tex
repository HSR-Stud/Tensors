

\documentclass[a4paper]{scrartcl}

% Mathematik-Pakete
\usepackage{amsmath}
\usepackage{amssymb}
\usepackage{amstext}
\usepackage{amsfonts}
\usepackage{mathrsfs}
\usepackage{paralist}
\usepackage[pdftex]{graphicx}
\usepackage{fancyhdr}

%\usepackage[ansinew]{inputenc}
%\usepackage[T1]{fontenc}

\usepackage[utf8]{inputenc}
%\usepackage[latin1]{inputenc}  
%\usepackage[T1]{fontenc}
%\usepackage{lmodern}          
\usepackage[a4paper]{geometry}
\geometry{verbose,tmargin=1.5cm,lmargin=2cm,rmargin=2cm,bmargin=2cm,nohead}
\usepackage[ngerman]{babel}


\begin{document}
\pagestyle{fancy}
\setlength{\footskip}{10mm}
\fancyhf{}
\renewcommand{\headrulewidth}{0pt}
\renewcommand{\footrulewidth}{0.5pt}

%\twocolumn[
 \centerline{\LARGE \bf \textsf{MSE CFD}} 
 \smallskip
\centerline{\Large \bf \textsf {Zusammenfassung}}
\medskip
  \centerline{\bf \textsf{dstrebel, s1bischo, sboller }}

 \smallskip \noindent\rule{\textwidth}{0.5pt}
\smallskip%]



\section{Juhui Tensoren...}

Gradient eines Skalarfeldes: gibt die Richtung und Stärke des steilsten Anstiegs
eines Skalarfeldes an.
Skalarfeld => Vektorfeld

Divergenz eines Vektorfeldes (Quellendichte eins Vektorfeldes): Gibt die Tendenz
eines Vektorfeldes an, von Punkten wegzufliessen (das gilt für positives
Vorzeichen; bei negativem Vorzeichen handelt es sich dementsprechend um die
Tendenz zu den Punkten hinzufliessen) Vektorfeld => Skalarfeld

Rotation eines Vektorfeldes: Gibt die Tendenz eines Vektorfeldes an, um Punkte
zu rotieren; die Rotation eines Vektorfeldes ist ein Vektorfeld von
Pseudovektoren. Und zwar folgt aus dem Stokes’schen Integralsatz (siehe unten),
dass die Rotation die lokale Wirbeldichte eines Vektorfeldes beschreibt.

Quelle: http://de.wikipedia.org/wiki/Vektoranalysis



\section{Description of Physical Phenomena}
\subsection{Einheiten}

$ [\rho]=\frac{kg}{m^3} $\\
$ [c_p]=\frac{J}{kgK} $\\
$ [m]=kg $\\
$ [T]=K $\\
$ [j_w]=\frac{W}{m^2} $\\
$ [\lambda]=\frac{W}{mK} $\\
$ [\nabla]=\frac{1}{m} $\\
$ [w]=\frac{kg}{ms^2} nochmals kontrollieren!!! $\\

\subsection{SI-Einheiten}
$ J = \frac{kg m^2}{s^2} $\\

\subsection{Begriffe}
\begin{itemize}
\item extensive Grösse: wenn sich System verdoppelt verdoppelt sich auch die Eigenschaft z.B. Impuls, Volumen ...
\item intensive Grösse: wenn sich System verdoppelt verdoppelt sich Eigenschaft nicht z.B. Druck, Temperatur, ...
\end{itemize}


\section{Bilanzgleichungen}

\begin{tabular}{|c|c|c|c|c|}
\hline  & Stoffliche Grösse & Potential & Balance Equation uniform & Balance Equation distributed \\ 
\hline Charge Transport & Charge Q & el. Potential ??? & $\frac{dQ}{dt}=I_Q$ & $ \frac{\parallel p_Q}{\partial t}=-\nabla\cdot j_Q$ \\ 
\hline Heat Transport &  &  &  &  \\ 
\hline Particle Transport &  &  &  & $\frac{dp}{dt}=-\nabla \cdot j_W +q_W$ \\ 
\hline Mechanics (Momentum Transport) & Momentum p & Velocity v  &  & ?? \\ 
\hline 
\end{tabular} 




\section{Vektoranalysis und Integraltheorem}
\subsection{Gradient}
Der Gradient ist ein mathematischer Operator, genauer ein Differentialoperator,
der auf ein Skalarfeld angewandt werden kann und in diesem Fall ein
Gradientenfeld genanntes Vektorfeld liefert, das die Änderungsrate d Richtung
der größten Änderung des Skalarfeldes angibt.

\begin{align}
\operatorname{grad}(f(x,y,z)) = \nabla f = \frac{{\partial f}}{{\partial x}} +
\frac{{\partial f}}{{\partial y}} + \frac{{\partial f}}{{\partial z}} 
\end{align}
\subsection{Divergenz}
In der Mathematik ist die Divergenz ein Differentialoperator, der einem
Vektorfeld ein Skalarfeld zuordnet. Während bei einem Vektorfeld jedem Punkt ein
Vektor zugeordnet wird, wird bei einem Skalarfeld jedem Punkt ein Skalar, also
eine Zahl, zugeodnet. Interpretiert man das Vektorfeld als Strömungsfeld, so
gibt die Divergenz für jeden Raumpunkt an, wie viel mehr aus einer Umgebung
dieses Punkts hinausfließt als in sie hineinfließt. Mithilfe der Divergenz lässt
sich also herausfinden, ob und wo das Vektorfeld Quellen (Divergenz größer als
Null) oder Senken (Divergenz kleiner als Null) hat. Ist die Divergenz überall
gleich Null, so bezeichnet man das Feld als quellenfrei.

\begin{align}
\vec F = \begin{pmatrix} F_1 \\ F_2 \\ F_3 \end{pmatrix}
\end{align}

\begin{align}
\operatorname{div}\vec F =
\frac{\partial}{\partial x_1}F_1 + \frac{\partial}{\partial x_2}F_2+ \frac{\partial}{\partial x_3}F_3
\end{align}

\subsection{Satz von Gauss}
Der Integralsatz von  liefert den Zusammenhang zwischen einem 
Volumenintegral über ein Volumen $V$, das von einem Feld $\vec F$ durchsetzt
ist, und einem Oberflächenintegral über die dieses Volumen umschliessende
Fläche $A$. Die Orientierung der Fläche sei so festgelegt, daß die Aussenseite
die positive Seite ist.

\begin{align}
\int_V \operatorname{div} \vec u \; \mathrm d^{(n)}V = \oint_{S} \vec u \cdot
\vec n\; \mathrm d^{(n-1)}A\,.
\end{align}
\subsection{Satz von Stokes}
Das Kurven- oder Linienintegral eines räumlichen Vektorfeldes $\vec u$ längs
einer geschlossenen Kurve $\partial F$ ist gleich dem Oberflächenintegral der
Rotation von $\vec u$ über eine beliebige Fläche $F$, die durch die Kurve
$\partial F$ berandet wird:

\begin{align}
 \int_{F} \operatorname{rot} \vec u \cdot d \vec A =
\oint_{\partial F} \vec u \cdot d \vec r
\end{align}
\end{document}




