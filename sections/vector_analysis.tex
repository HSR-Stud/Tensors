
		\subsection{General Remarks}		
		\textcolor{blue}{\textbf{Gradient}}: 

		\textbf{Direction, Angle}: points to the steepest ascent.
		
		\textbf{Absolute Value}: is the greater, the stronger a value changes at a position.
		
		\textbf{scalar field}$\Rightarrow$ \textbf{vector field}
				
		\textcolor{blue}{\textbf{Divergence}}:
		
		Gives the tendencie of a vector field, to flow away from a point.
		
		$\operatorname{div}\vec F(\vec r) = q(\vec r) \begin{cases}
			> 0, & \text{source, quelle}\\
			= 0, & \text{source free}\\
			< 0, & \text{sink, senke}
		\end{cases}$
		
		\textbf{Vector field} $\Rightarrow$ \textbf{scalare field}
		
		\textcolor{blue}{\textbf{Curl, Rotation}}
		
		Tendency to rotate around a point. 
		
		The rotation of a vector field is a vector field of pseudo vectors. 
		
		Describes the local curl density of a vector field.		
		
		\textbf{vector field}$\Rightarrow$ \textbf{vector field}

		\textcolor{blue}{\textbf{signs and notation}}: 
		
		\begin{tabularx}{\linewidth}{l|X}
		$\text{Nabla operator:}\quad \nabla \: \mathrm{oder} \: \vec \nabla = \left (\dfrac\partial{\partial
			x_1},\ldots, \dfrac\partial{\partial x_n}\right) $ 
		&
		$\text{Laplace operator:}\quad \laplace \: \mathrm{oder} \: \vec \laplace = \left (\frac{\partial^2}{\partial
			x_1^2},\ldots, \frac{\partial^2}{\partial x_n^2}\right)$
		\\
		\hline
		$\text{Scalar function:}\quad f = f(x,y,z)$
		&
		$\text{vector field:}	\quad \vec F = \begin{pmatrix} F_1 \\ F_2 \\ F_3 \end{pmatrix}$
		\\
		\hline
		\multicolumn{2}{l}{
		$
		\text{Vector product in cartesian coordinates:}\quad\vec{a}\times\vec{b} = \begin{vmatrix}
		\textbf{i} & \textbf{j} & \textbf{k} \\ a_1 & a_2 & a_3 \\ b_1 & b_2 & b_3\\ \end{vmatrix} =
		\begin{pmatrix}a_1 \\ a_2 \\ a_3\end{pmatrix} \times
		\begin{pmatrix}b_1 \\ b_2 \\ b_3 \end{pmatrix} = \begin{pmatrix} a_2b_3 - a_3b_2
		\\ a_3b_1 - a_1b_3 \\ a_1b_2 - a_2b_1 \end{pmatrix}$}\\
	\hline 
	\end{tabularx}
			
\renewcommand{\arraystretch}{2}		
		\begin{tabularx}{\linewidth}{lllllX}
			\hline
			Function & &Nabla & &Differential form & Note \\
			\hline
			$\operatorname{grad}(f)$ &=& 
			$\nabla f$ &=&
			 $ 
			\begin{pmatrix} \dfrac{\partial f}{\partial x}, \quad\dfrac{\partial f}{\partial
				y}, \quad \dfrac{\partial f}{\partial z} \end{pmatrix}^T$&\\	
			\hline 
			$\operatorname{div}\vec F$ &=& 
			$\nabla \cdot \vec F$&=&
			$\dfrac{\partial}{\partial x_1}F_1 + \dfrac{\partial}{\partial x_2}F_2+ \dfrac{\partial}{\partial x_3}F_3$&
			$\operatorname{div}\vec F = \dfrac{\sum_i I_i}{V}$\\
			\hline
			$\mathbf{\operatorname{rot}}\,\vec F$ &=& 
			$\nabla\times \vec F$ &=& 
			$\begin{pmatrix} \dfrac{\partial}{\partial x} \\ \dfrac{\partial}{\partial y} \\
				\dfrac{\partial}{\partial z} \end{pmatrix} \times \begin{pmatrix} F_x\\ F_y\\ F_z
			\end{pmatrix} = \begin{pmatrix} \dfrac{\partial F_z}{\partial y} - \dfrac{\partial
				F_y}{\partial z} \\ \dfrac{\partial F_x}{\partial z} - \dfrac{\partial
				F_z}{\partial x} \\ \dfrac{\partial F_y}{\partial x} - \dfrac{\partial
				F_x}{\partial y} \end{pmatrix}$&
			$\mathbf{\wedge = \times} \newline$\\ 
			&&
			&=&
			$ \left (\dfrac{\partial F_y}{\partial x} - \dfrac{\partial F_x}{\partial y}\right )\vec e_z$ & 2 Dimensional\\
			\hline
		\end{tabularx}

\renewcommand{\arraystretch}{1}				


	
	\subsection{Calculation rules (Bronstein p.731)}
	
{\rowcolors{3}{lightgray}{white}	
\renewcommand{\arraystretch}{1.2}					
	\begin{tabularx}{\textwidth}{|X|X|}
		\hline 
		\multicolumn{2}{|l|}{
		Vector filds : $\vec F$, $\vec G$  $\qquad$ 
		Scalar fields: $U$, $V$	           $\qquad$ 
		constant, constant field: $c, \vec{c}$}\\			
		
		\hline
		$\operatorname{grad}\,c=\vec{0}$ & $\operatorname{grad}\,(c\cdot u)=c\cdot\operatorname{grad}\,U$    \\

		$\operatorname{grad}\,(U+V)=\operatorname{grad}\,U+\operatorname{grad}\,V$ &
		$\operatorname{grad}\,(U\,V) = U\ \operatorname{grad}\,V + V\ \operatorname{grad}\,U$\\

		$\label{eqn:ProduktregelEinesGradientMitPotenzen}
		\operatorname{grad}\,(U^n) = n\, U^{n-1}\ \operatorname{grad}\,U \text{ für } n\neq 0$&
		$\operatorname{rot}(\operatorname{grad}U)=\nabla \times (\nabla U) = 0$ \\
		\hline\hline
		$\operatorname{div}\,\vec{c}=\vec{0}$&
		$\operatorname{div}\,(c\cdot \vec{F})=c\cdot\operatorname{div}\,\vec{F}$ \\
		
		$\operatorname{div}\,(\vec{F}+\vec{G})=\operatorname{div}\,\vec{F}+\operatorname{div}\,\vec{G}$ &
		$\operatorname{div}\,(U\,\vec{F}) = U\ \operatorname{div}\,\vec{F} + \vec{F}\cdot \operatorname{grad}\,U$\\		

		$\operatorname{div}\,(\vec{F} \times\vec{G})=\vec{G}\cdot \operatorname{rot}\,\vec{F} - \vec{F}\operatorname{rot}\,\vec{G}$ &
		$\operatorname{div}(\operatorname{rot}\vec{F}) = \nabla \cdot (\nabla \times\vec F) = 0$\\
		\hline\hline
		
		$\operatorname{rot}\,(\vec{F}+\vec{G})=\operatorname{rot}\,\vec{F}+\operatorname{rot}\,\vec{G}$ &
		$\operatorname{rot}\,(c \vec{F})=c\operatorname{rot}\,\vec{F}$ \\

		$\operatorname{rot}\,(U\,\vec{F}) = U\ \operatorname{rot}\,\vec{F} + \operatorname{grad}\,U \times \vec{F}$&
		$\operatorname{rot}\,(\vec{F} \times\vec{G}) =\newline (\vec{G}\cdot \operatorname{grad})\,\vec{F} - (\vec{F}\cdot \operatorname{grad})\,\vec{G} + \vec{F}\operatorname{div}\vec{G} - \vec{G}\operatorname{div}\,\vec{F}$
		\\
		\hline\hline
		$	\operatorname{rot}(\operatorname{rot}\vec{F}) = \operatorname{grad}(\operatorname{div}\vec{F}) -\Delta \vec{F}$&\\

		$\operatorname{grad}(UV)=U\operatorname{grad}V+V\operatorname{grad}U $ &
		$\operatorname{grad}(\vec{F}\cdot \vec{G}) = (\operatorname{grad}\vec{F})^{\operatorname t}\vec{G} + (\operatorname{grad}\vec{G})^{\operatorname t}\vec{F} $ \\

		$	\operatorname{div}(U\vec{F})=U\operatorname{div}\vec{F}+\vec{F}\cdot\operatorname{grad}U $&
		$\operatorname{div}(\vec{F}\times \vec{G})= \vec{G} \cdot\operatorname{rot}\vec{F} -\vec{F}\cdot\operatorname{rot}\vec{G} $ \\				

		$\operatorname{rot}(U\vec{F})= U\operatorname{rot}\vec{F}-\vec{F}\times\operatorname{grad}U\ $& 
		$\operatorname{grad}\,\varphi(U) = \dfrac{d\varphi}{dU}\operatorname{grad}\,U$
		\\
		\hline
	
		
	\end{tabularx}
}
\renewcommand{\arraystretch}{1.2}		

	\paragraph{Potential Field}
		If a vector field is path independant it is called conservative. This vector field is the gradient of a potential function $\phi$
		
		\begin{multicols}{3}
			$$ -\int\limits_a^b \vec F\cdot d\vec x = \phi(\vec b) - \phi(\vec a) $$ \vfill\columnbreak
			$$ \oint \vec F\cdot d\vec x = 0 $$ 
			$$ \nabla \times \vec F = 0  $$ \vfill\columnbreak
			$$ F = \operatorname{grad}\phi $$
		\end{multicols}
		
		
	\paragraph{Line integral}~\\
	\begin{tabularx}{\columnwidth}{lp{2.5cm}p{4cm}X}
	\hline 
		\multicolumn{4}{c}{receipt to Line integrals in a vector field}\\
	\hline 
		1& Partition & Divide the curve into pieces & $I = \int\limits_C\bm fd\bm r = \int\limits_{C_1}\bm fd\bm r_1 + \int\limits_{C_2}\bm fd\bm r_2+\dots = \sum\limits_i \int\limits_{C_i}\bm fd\bm r_i$\\
		2& Line vector& 
		 Describe the curve $C_i$ as a function of t & $\bm r_i(t) = \left(x_i(t), y_i(t), z_i(t)\right)^T\quad t\in[a,b]$\\
		3& Derivative & take the derivative of $\bm r_i(t)$ & $\bm r_i'(t) = \left(x_i'(t), y_i'(t), z_i'(t)\right)^T \quad d\bm r_i = \bm r_i'(t) dt $\\
		4& parametrize the field  & put $\bm r_i(t)$ into the vector field function & $\bm f(x,y,z) = \bm f(\bm r_i(t) ) = \bm f(x_i(t), y_i(t), z_i(t))$\\
		5& integrals & calculate the integrals & $I_i = \int\limits_{C_i} = \bm f(\bm r_i(t))\cdot \bm r_i'(t)dt$\\
		6& sum up & summarize the pieces & $I = \sum\limits_i I_i$\\
		\hline 
	\end{tabularx}
	scalar field $$I = \int\limits_C f d \bm r = \int\limits_{a}^b f(\bm r(t))\; |\bm r'(t)| dt$$
	
	vector field $$I = \int\limits_C \bm f d\bm r = \int\limits_{a}^b \bm f(\bm r(t))\; \cdot \bm r'(t) dt$$
		
	\paragraph{special vectors}  
	$\bm r = \begin{pmatrix} x\\y\\z\end{pmatrix} \qquad |\bm r| = r = \dfrac{1}{\sqrt{x^2+y^2+z^2}}$
	
	$\nabla\left(\dfrac{1}{r}\right) = -\dfrac{\bm r}{r^3}\qquad
	\nabla\left(\dfrac{1}{r^2}\right) = -\dfrac{2\bm r}{r^4}\qquad
	\nabla\left(\dfrac{1}{r^3}\right) = -\dfrac{3\bm r}{r^5}\qquad
	\nabla (\bm r \bm F) = \bm F$

	\subsection{Integral theorems}
	
		
		\textcolor{blue}{\textbf{Divergence Theorem of Gauss}}: Volume integral: Divergence of a Vectorfield $\Rightarrow$ Surface integral: Vectorfield\\
		
		$$\int_V \operatorname{div} \vec{F} \; \mathrm dV = 
				\int_V \nabla\cdot \vec{F} \; \mathrm dV =
				\oint_{S} \vec F \cdot\vec n\; \mathrm dA
		$$

	\textcolor{blue}{\textbf{Stoke's Theorem}}: Surface integral: rotation of a vector field $\Rightarrow$ line integral along the surface: vector field  
	
	$$\int_{F} \operatorname{rot} \vec F \cdot d \vec A =
		\int_{F} \nabla \times \vec F \cdot d \vec A =
		\oint_{\partial F} \vec F \cdot d \vec r $$
	


	
