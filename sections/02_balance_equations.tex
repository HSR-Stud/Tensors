	\subsection{Begriffe}
	\begin{multicols}{2}
		\begin{itemize}
			\item Gr�ssen
			\begin{itemize}
				\item \emph{Extensive / Substance like } Gr�ssen: wenn sich System verdoppelt verdoppelt sich auch die Eigenschaft z.B. Impuls, Volumen ...    
				\item \emph{Intensive} Gr�sse: wenn sich System verdoppelt verdoppelt sich Eigenschaft nicht z.B. Druck, Temperatur, ...
			\end{itemize}
			\vfill\columnbreak
			\item Randbedingungen
			\begin{itemize}
				\item Neumann: constant temperature
				\item Dirichlet: constant gradient
			\end{itemize}
		\end{itemize}
	\end{multicols}
	
	
	\subsection{Balance equation, Bilanzgleichungen}
	\begin{multicols}{2}
		Diese haben die Form :	
		$\dfrac{dG}{dt} = \sum\limits_{i} I_{G,i} + Q_G$
		
		\columnbreak 
		\begin{itemize}
			\item G, I und Q sind dabei extensive Gr�ssen
			
			\item $\dfrac{dG}{dt}$: Zeitliche �ndrung
			
			\item $I_X$: Fluss
			
			\item $Q_G$: Quellenterm: Prozess, der im System abl�uft
		\end{itemize}
	\end{multicols}
	
	
	{
		\begin{tabularx}{\linewidth}{XXXcc}
			\toprule
			& Stoffliche Gr�sse & Potential & \multicolumn{2}{c}{Balance Equation} \\
			& & & uniform & distributed, local \\ \midrule
			Charge Transport & Charge $Q$ & El. Potential $\Phi$ & $\frac{dQ}{dt}=I_Q$ & $ \frac{\partial p_Q}{\partial t}=-\nabla\cdot j_Q$ \\ \hline
			Heat Transport & Thermal energy \newline $W$ & Temperatur $T$ & $\frac{\partial W}{\partial t}= I_W+Q_W$ & $\frac{\partial w}{\partial t}= -\nabla \cdot j_W+q_W$  \\ \hline
			Particle Transport & Partical number \newline $N$ & Particle density \newline $c$ & $\frac{dN}{dt}=I_N+Q_N$ & $\frac{dp}{dt}=-\nabla \cdot j_W +q_W$ \\ \hline
			Mechanics (Momentum Transport) & Momentum $p$ & Velocity $v$  & $\frac{\partial p}{\partial t}=F$ & ?? \\ 
			\bottomrule
		\end{tabularx} }
		
		\subsection{Constitutive Laws}
		Folgt aus den Bilanzgleichungen. Das System wird auf physikalische Umst�nde festgelegt.
		
		\begin{tabularx}{\linewidth}{Xcp{4cm}X}
			\toprule
			& Storage Laws & Constitutive Laws & Source Rates and Sourcerate Densities \\ \midrule
			Charge Transport & & $j_Q=-\sigma \cdot \nabla\varphi \equiv \sigma \cdot E$ \newline (Microscop. Ohmic law) & $q_+ = - q_-$ \\
			Heat Transport & $w=c_p \rho T$ & $j_W=-\lambda \cdot \nabla T$  (Fourier's law)& $Q_W = -(\phi_2-\phi_1)I_Q$\newline $q_W = -\nabla\Phi\cdot j_Q$  \\ 
			Particle Transport (Diffusion) & c=c & $j_N=- D \cdot \nabla c_N$ &  \\ 
			Mechanics & $p=mv$ & $F_{hook} = - k \Delta u$ (Hook's Law)\newline $F_{visc}=-C \eta \cdot v$ (Viscous Drag) & $F_{grav} = M g$\newline $f_{grav}=\rho g$ \\ 
			\bottomrule
		\end{tabularx} 
		\begin{multicols}{2}
			\emph{Isotropic/Anisotropic:} An isotropic relation is independent of the spatial direction of the gradient (richtungsunabh�ngig)
			
			\columnbreak
			
			\emph{Homogeneous/Inhomogeneous:}
			if a body is spatially constant, its homogeneous (material besitzt �berall gleiche werte).
			
			\emph{Linear/Nonlinear:} balance equation are always linear, consecutive laws can be nonlinear. 
			
		\end{multicols}
		\subsection{Beispiele}
		\subsubsection{Ohm's Law}
		% Exercise Chapter 1, Aufgabe 2
		...bitte noch �bertragen...
		\subsubsection{Heat conduction in 1D}
		% Exercise Chapter 1, Aufgabe 3
		\begin{itemize}
			
			\item Formulate the stationary het balance equation for a 1D system (slender rod). Genommen wird die Allgemeine W�rmeleitungsgleichung $\frac{\partial w}{\partial t}= -\nabla \cdot j_W+q_W$ und diese Vereinfacht, da Station�r und 1D ergibt es $0= -\frac{\partial j_W}{\partial x}+q_W$
			
			\item Formulate Fourier's law of heat conduction in 1D. Dies ist Allgemein $j_W=-\lambda \cdot \nabla T$ und im 1D $j_W=-\lambda \frac{\partial T}{\partial x} T$
			
			\item Solve this system of 2 ODEs of $1^{st}$ order for following conditions: a Al rod of length L=20cm , cross section $A=1cm^2$ and specific heat conductivity 
			$ \lambda = 240 W/(Km) $ 
			is cooled to $25^\circ C$ on its right hand side and heated on the lhs with a heat flux density of $ 36 kW/m^2 $ (candle).
			Es gilt das Fourier's law in die balance equation einzusetzen ergibt $ 0=\lambda \frac{\partial^2T}{\partial x^2}+q_W$, da es keine Quelle oder Senke hat gilt $q_W=0$. L�st man diese Differentialgleichung ergibt es $T(x)=C_1x+C_2$. An der Stelle (x=0) gilt das Fourier's law, also gilt $j_W=-\lambda \frac{\partial T}{\partial x} = 36000 = -240 \frac{\partial T}{\partial x } \Rightarrow \frac{\partial T}{\partial x}=-150 \frac{K}{m} =C_1$ und auch gegeben ist $25^\circ C$ an der Stelle x=0.2m. Damit $C_2$ ausrechnen ergibt  $ T(0.2)= -150 \frac{K}{m} \cdot 0.2m+C_2=25^{\circ} C $. Allgemein ergibt sich die Gleichung $ T(x)= -150 \frac{K}{m} \cdot x [m] + 25^{\circ} C $.
			
			\item Neu sind beide Seiten sind $25^\circ K$ Nun kommt zus�tzlich noch eine W�rmequellerate f�r $500 kW/m^3$ nun gibt es $ -\lambda \frac{\partial^2T}{\partial x^2}=q_W$ l�st man diese Partielle DGL ergibt es $T(x)=\frac{1}{2}\frac{-q_W}{\lambda} x^2+C_1x+C_2$ Setzt man die Randbedingungen ein ergibt es $T(x)=\frac{1}{2}\frac{q_W}{\lambda}(Lx-x^2)+T_0$
			
			
		\end{itemize}

}%rowcolor	tabularx				
