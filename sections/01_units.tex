	\subsection{Einheiten}
	
	\begin{multicols}{2}
		\paragraph{Mathematics} ~\\

		\begin{tabularx}{\linewidth}{lclX}
			$[\nabla]$ &=& \si{1\per\meter} & nabla\\
			$[\laplace]$ &=& \si{1\per\meter\tothe{2}} & laplace\\
		\end{tabularx}

		\paragraph{General}~\\
		\begin{tabularx}{\linewidth}{lclX}
			$[F]$ &=& \si{\newton=\kilogram\meter\per\second\tothe{2}}& Kraft, force \\
			$[E]$ &=& \si{\joule=\kilogram\meter\tothe{2}\per\second\tothe{2}}& Energie, energy \\
			$[P]$ &=& \si{\joule\per\second=\kilogram\meter\tothe{2}\per\second\tothe{3}}& Leistung, power \\
			$[p]$ &=& \si{\pascal= \newton\per\meter\tothe{2}}& Druck, pressure \\
			$[V]$ &=& \si{\meter\tothe{3}}& Volumen, volume\\
			$[n]$ &=& \si{\mole}& Stoffmenge, Amount of substance\\
			$[T]$ &=& \si{\kelvin}& Temperatur, temperature \\
			$[m]$ &=& \si{\kilogram} & Masse, mass \\
			$[\rho]$ &=& \si{\kilogram\per\meter\tothe{3}} & Dichte, mass density \\
			$[A]$ &=& \si{\meter\tothe{2}} & Fl�che, surface\\
		\end{tabularx}
		\paragraph{Heat}~\\
		\begin{tabularx}{\linewidth}{lclX}
			$[Q_W]$ &=& \si{\joule} & W�rme, heat\\
			$[q_\omega]$ &=& \si{\joule\per\meter\tothe{3}} & Thermische Leitf�higkeit, thermal energy source rate density \\	
			$[W]$ &=& \si{J} & W�rmeenergie, thermal energy\\
			$[\omega]$ &=& \si{\joule\per\meter\tothe{3}} & Energiedichte, thermal energy storage density \\
			$[I_W]$ &=& \si{\joule\per\second=\watt} & W�rmefluss, heat flux\\
			$[j_w]$ &=& \si{\watt\per\square\meter} & Energieflussdichte, thermal energy flux density \\
			$[c_p]$ &=& \si{\joule\per\kilogram\kelvin} & spezifischer W�rmekapazit�t, specific heat capacity\\
			$[\lambda]$ &=& \si{\watt\per\meter\kelvin} & Thermische Leitf�higkeit \\	
		\end{tabularx}
		
		\paragraph{Mechanics}~\\
		\begin{tabularx}{\linewidth}{lclX}
			$[p]$ &=& \si{\newton\second} & Impuls \\
			$[J_p]$ &=& \si{\newton} & Impulsfluss, Kraft, Force \\
			$[j_p]$ &=& \si{\newton\per\square\meter} = \si{\pascal} & Impulsflussdichte\\
			$[\sigma]$ &=& \si{\newton\per\square\meter} & Spannung (stress), neg. Impulsflussdichte \\
			$[W]$ &=& \si{\newton\meter} & arbeit, work, mechanical energy\\
			$[\varepsilon l]$ &=& 1& Dehung (Strain)\\
			$[l_0]$ &=& \si{m} & Urspr�ngliche L�nge \\
			$[\Delta l]$ &=& \si{m} & L�ngenausdehnung \\
			$[d]$ &=& \si{\newton\per\meter} & Federkonstant, Spring constant\\
		\end{tabularx}
		
		\paragraph{Particle transport, Stofftransport}~\\
		\begin{tabularx}{\linewidth}{lclX}
			$[c_n]$ &=& \si{\mole\per\meter\cubed} & particle density concentration\\
			$[I_n]$ &=& \si{\mole\per\second} & particle flux\\
			$[j_n]$ &=& \si{\mole\per\meter\squared\second} & particle flux density\\
			$[D_n]$ &=& \si{\meter\squared\per\second} & Diffusionskoeffizient, diffusion coefficient\\
		\end{tabularx}
			
		\paragraph{Masstransport, Massentransport}~\\
		\begin{tabularx}{\linewidth}{lclX}
			$[M]$ &=& \si{\kilogram} & masse\\			
			$[I_{M}]$ &=& \si{\kilogram\per\second} & Massenfluss, mass transport, mass flux\\			
			$[j_M]$ &=& \si{\kilogram\per\meter\squared\second} & Massenflussdichte, mass flux density\\
		\end{tabularx}
	\end{multicols}
	
	\paragraph{Electricity, Elektrotechnik}~\\
	
	\begin{tabularx}{\linewidth}{lclX}
		\multicolumn{4}{c}{\textbf{Electrical network}}\\
		$[U]$ &=& \si{\volt} = \si{\newton\meter\per\ampere\second} & Elektrische Spannung, voltage \\
		$[I]$ &=& \si{\ampere} & Elektrischer Strom, current \\

		$[R]$ &=& \si{\ohm} = \si{\kilogram\square\meter\per\square\ampere\second\tothe{3}} & Elektrischer Widerstand, resistance \\
		$[G]$ &=& \si{\siemens} = \si{\per\ohm} & Elektrische Leitf�higkeit, conductivity \\
		$[\rho]$ &=& \si{\ohm\meter = \kilogram \meter\cubed \per\ampere\tothe{3}\second\tothe{3}} & Spezifischer Widerstand, specific resistance \\
		$[\sigma]$ &=& \si{\per\ohm\meter} = \si{\ampere\per\volt\meter} = \si{\siemens\per\meter} & Spezifische Leitf�higkeit, specific conductivity \\
		$[W]$ &=& \si{\joule} &Electric Energy\\
		$[\dot W]=[P]$ &=& \si{\watt} = \si{\joule\per\second} & Elektrische Leistung, power \\
		
				
		%elektrostatik		
		\multicolumn{4}{c}{\textbf{Electrostatics}}\\
		$[Q]=[q]$ &=& \si{\coulomb} = \si{\ampere\second} & Elektrische Ladung, charge \\
		$[\rho]$ &=& \si{\ampere\second\per\meter\cubed} & charge density\\
		$[j]$ &=& \si{\ampere\per\square\meter} & Stromdichte, current density \\	
		$[\vec{E}]$ &=& \si{\volt\per\meter} = \si{\newton\per\coulomb} & Electric Field \\
		$[\phi]$ &=& \si{\volt} = \si{\kilogram\meter\squared\per\ampere\second\cubed} & Electrostatic potential \\
		$[\vec{p}\:]$ &=& \si{\coulomb\meter} & Electrical dipole moment \\
		$[\Phi]$ &=& \si{\volt\meter} & Electric Flux\\
		$[\vec{D}]$ &=& \si{\coulomb\per\square\meter} & Elektrische Flussdichte, Electric Displacement Field\\
		$[P]$ &=& \si{\coulomb\per\square\meter} & Polarisation Density \\
		$[\chi]$&=& - &Electric susceptibility\\
		$[\varepsilon_0]$ &=& \si{\ampere\second\per\volt\meter} = \si{\square\coulomb\per\newton\square\meter} & Elektrische Feldkonstante \\
		$[C]$ &=& \si{\ampere\second\per\volt} & Capacity \\
		$[w_E]$ &=& \si{\watt\second\per\meter\cubed} = \si{\joule\per\meter\cubed} & energy density\\
						
		%electro dynamics
		\multicolumn{4}{c}{\textbf{Electrodynamics}}\\
		

		
		$[B]$ &=& \si{\tesla} (Tesla) = \si{\kilogram\per\ampere\square\second} = \si{\newton\per\ampere\meter} = \si{\volt\second\per\square\meter} & Magnetische Flussdichte, conductance \\
		$[H]$ &=& \si{\ampere\per\meter} & Magnetische Feldst�rke \\
		
		$[w]$ &=& \si{\kilogram\per\meter\square\second=\pascal} & \colorbox{red}{nochmals kontrollieren!!!} \\
		
		$[M]$ &=& \si{\newton\meter} & Drehmoment \\

		$[m]$ &=& ??? & Magnetic Moment \\

		$[\mu_0]$ &=& \si{\volt\second\per\ampere\meter} & Magnetische Feldkonstante \\
		$[k]$ &=& \si{\volt\meter\per\ampere\second} & Coulomb-Konstante \\
	\end{tabularx}
	